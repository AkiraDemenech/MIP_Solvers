\documentclass[]{article}
\usepackage{amsfonts} 

%opening
\title{}

\begin{document}

\maketitle

\begin{abstract}

\end{abstract}

\section{Modelagem dos problemas}

	Consideramos dois casos do problema de localização de facilidades com capacidade limitada 
	(\textit{Capacitated Facility Location Problem}, CFLP):
	com fonte única (\textit{Single Source}, SS) 
	e com múltiplas fontes. 
	
	\subsection{Problema de localização de facilidades com capacidade limitada e fonte única}
	% http://old.math.nsc.ru/AP/benchmarks/CFLP/cflp-eng.html
	
	No caso de fonte única, a limitação de capacidade é um valor só $s$ fixado para todas as facilidades.
	O custo fixo $f$ de abertura também é o mesmo para todas.
	O conjunto de facilidades é dado por $I$ e o de clientes por $J$.
	A formulação adotada traz a demanda $p_{ij}$ do cliente $j \in J$ se for atendido pela facilidade $i \in I$, 
	sendo possível que o cliente $j$ não possa ser atendido pela facilidade $i$.
	O custo de transporte $g_{ij}$ da facilidade $i \in I$ para o cliente $j \in J$ é referente à toda a demanda, não ao transporte de cada unidade (ou medida) requerida.
	$x_{ij}$ indica se a facilidade $i \in I$ atenderá a demanda do cliente $j \in J$.
	A variável binária $y_i$ indica se a facilidade $i \in I$ será aberta ou não.
	
	São aplicadas as restrições de capacidade das facilidades (\ref{ss:const:capacity}) e de satisfação da demanda (\ref{ss:const:demand}).
	
	\begin{equation}
		\label{ss:const:capacity}		
		\sum_{j \in J} x_{ij} p_{ij} \le y_i s 
		\quad
		\forall i \in I
	\end{equation}	

	\begin{equation}
		\label{ss:const:demand}		
		\sum_{i \in I} x_{ij} \ge 1 
		\quad
		\forall j \in J
	\end{equation}

	As variáveis devem ser binárias (\ref{ss:const:dom}) 
	e o objetivo é minimizar os custos de abertura e transporte (\ref{ss:obj}).
	
	\begin{equation}
		\label{ss:const:dom}		
		x_{ij}, y_i \in \{0, 1\}
		\quad
		\forall i \in I, j \in J
	\end{equation}	
	
	\begin{equation}
		\label{ss:obj}		
		\min \sum_{i \in I} 
		(
			f y_i + \sum_{j \in J} g_{ij} x_{ij}
		)
	\end{equation}

	A relaxação linear das variáveis $x$ (da forma $x_{ij} \in [0,1]$) transformaria esse caso em um problema com múltiplas fontes, 
	o modelo, porém, se tornaria bastante estranho:  
	demandas $p_{ij}$ diferentes poderiam ser parcialmente atendidas, satisfazendo uma demanda mista não-planejada.
	
	\subsection{Problema de localização de facilidades com capacidade limitada e múltiplas fontes}
	% https://link.springer.com/article/10.1007/s00500-022-07600-z 
	
	No caso de fontes múltiplas, para o conjunto de facilidades $I$ e de clientes $J$, 
	a capacidade $s_i$ e o custo fixo de abertura $f_i$ não são necessariamente os mesmos para todas as facilidades $i \in I$, 
	enquanto a demanda $d_j$ do cliente $j \in J$ é a mesma
	independente de qual (ou quais) facilidade(s) a satisfaça(m). 
	O custo de transporte $c_{ij}$, por unidade, da facilidade $i \in I$ para o cliente $j \in J$ existe para todos os pares.
	No modelo utilizado, é introduzido o conjunto $\Gamma$ de pares de clientes incompatíveis $\langle i_1, i_2 \rangle \in \Gamma$
	
	Da mesma forma que o caso anterior, são aplicadas as restrições de capacidade das facilidades (\ref{ms:const:capacity}) e de satisfação da demanda (\ref{ms:const:demand}).
	
	\begin{equation}
		\label{ms:const:capacity}		
		\sum_{j \in J} x_{ij} d_{j} \le y_i s_i 
		\quad
		\forall i \in I
	\end{equation}	
	
	\begin{equation}
		\label{ms:const:demand}		
		\sum_{i \in I} x_{ij} \ge 1 
		\quad
		\forall j \in J
	\end{equation}
	
	As variáveis de abertura devem novamente ser binárias, enquanto o atendimento deve ser real (\ref{ms:const:dom}) 
	e o objetivo é minimizar os custos de abertura e transporte (\ref{ms:obj}).
	
	\begin{equation}
		\label{ms:const:dom}		
		y_i \in \{0, 1\}, 
		x_{ij} \ge 0, x_{ij} \in \mathbb{R}
		\quad
		\forall i \in I, j \in J		
	\end{equation}	
	
	\begin{equation}
		\label{ms:obj}		
		\min \sum_{i \in I} 
		(
		f y_i + \sum_{j \in J} g_{ij} x_{ij}
		)
	\end{equation}

\end{document}
