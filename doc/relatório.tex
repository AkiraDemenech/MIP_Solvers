\documentclass[]{article}

%opening
\title{}

\begin{document}

\maketitle

\begin{abstract}

\end{abstract}


\section{Modelagem dos problemas}

	Consideramos dois casos do problema de localização de facilidades com capacidade limitada 
	(\textit{Capacitated Facility Location Problem}, CFLP):
	com fonte única (\textit{Single Source}, SS) 
	e com múltiplas fontes. 
	
	\subsection{Problema de localização de facilidades com capacidade limitada e fonte única}
	% http://old.math.nsc.ru/AP/benchmarks/CFLP/cflp-eng.html
	
	No caso de fonte única, a limitação de capacidade era um valor só $c$ fixado para todas as facilidades.
	O custo fixo $f$ de abertura também era o mesmo para todas.
	O conjunto de facilidades é dado por $I$ e o de clientes por $J$.
	A formulação adotada traz a demanda $p_{ij}$ do cliente $j \in J$ se for atendido pela facilidade $i \in I$, 
	O custo de transporte $g_{ij}$ da facilidade $i \in I$ para o cliente $j \in J$ é referente à toda a demanda, não ao transporte de cada unidade (ou medida) requerida.
	A variável binária $y_i$ indica se a facilidade $i \in I$ será aberta ou não.
	
	

\end{document}
