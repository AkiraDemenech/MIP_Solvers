\documentclass[]{article}

%opening
\title{}

\begin{document}

\maketitle

\begin{abstract}

\end{abstract}

\section{Modelagem dos problemas}

	Consideramos dois casos do problema de localização de facilidades com capacidade limitada 
	(\textit{Capacitated Facility Location Problem}, CFLP):
	com fonte única (\textit{Single Source}, SS) 
	e com múltiplas fontes. 
	
	\subsection{Problema de localização de facilidades com capacidade limitada e fonte única}
	% http://old.math.nsc.ru/AP/benchmarks/CFLP/cflp-eng.html
	
	No caso de fonte única, a limitação de capacidade é um valor só $c$ fixado para todas as facilidades.
	O custo fixo $f$ de abertura também é o mesmo para todas.
	O conjunto de facilidades é dado por $I$ e o de clientes por $J$.
	A formulação adotada traz a demanda $p_{ij}$ do cliente $j \in J$ se for atendido pela facilidade $i \in I$, 
	sendo possível que o cliente $j$ não possa ser atendido pela facilidade $i$.
	O custo de transporte $g_{ij}$ da facilidade $i \in I$ para o cliente $j \in J$ é referente à toda a demanda, não ao transporte de cada unidade (ou medida) requerida.
	$x_{ij}$ indica se a facilidade $i \in I$ atenderá a demanda do cliente $j \in J$.
	A variável binária $y_i$ indica se a facilidade $i \in I$ será aberta ou não.
	
	São aplicadas as restrições de capacidade das facilidades \ref{ss:const:capacity} e de satisfação da demanda \ref{ss:const:demand}.
	
	\begin{equation}
		\label{ss:const:capacity}		
		\sum_{j \in J} x_{ij} p_{ij} \le y_i c 
		\;
		\forall i \in I
	\end{equation}	

	\begin{equation}
		\label{ss:const:demand}		
		\sum_{i \in I} x_{ij} \ge 1 
		\;
		\forall j \in J
	\end{equation}

\end{document}
