% Título do projeto: Avaliação de resolvedores para um problema clássico de otimização.
% Palavras chaves: Programação linear inteira mista, modelagem, softwares de otimização.

% Poderá ter até 5000 (cinco mil) palavras, considerando todas as seções do trabalho, inclusive título, resumo, referências e outras.

\documentclass[]{article}
\usepackage{hyperref}
\usepackage{amsfonts} % number sets in math mode 
\usepackage{textcomp} % TM
\usepackage{multicol} 


% required for the tables:
\usepackage{multirow}
\usepackage{rotating}

\usepackage{graphicx}
% pasta das imagens
\graphicspath{ {./img/} }

\newtheorem{observation}{Observação}

\newcommand{ \solver }[1]{\textit{#1}} 
\newcommand{ \python }[1]{\textit{#1}}
\newcommand{ \instance }[1]{\texttt{#1}}

% seções do relatório 
\newif\ifrelatorio
\relatoriotrue

% seções do artigo 
\newif\ifartigo
\artigotrue

%opening
\title{Avaliação de resolvedores para problemas de localização de facilidades com capacidade limitada}
\author{Guilherme Akira Demenech Mori}

% https://github.com/akirademenech/MIP_Solvers/blob/main/doc/relat%C3%B3rio.tex

\begin{document}

	\maketitle

	\begin{abstract}
	
	\end{abstract}

\ifrelatorio
	\tableofcontents		
	\ifartigo
		\listoftables
	\fi	
\fi

\ifartigo
	\section{Introdução}
\fi	
	
\ifrelatorio	
	\section{Atividades desenvolvidas}
	
		Os estudos teóricos foram guiados pelos capítulos de Otimização Linear e Discreta de \textit{Pesquisa operacional} \cite{arenales}.	
		Para os testes computacionais foi utilizada a linguagem de \textit{script} \python{Python} \cite{Python}, sendo feita leitura de instâncias e chamada dos resolvedores escolhidos.
		
		\begin{enumerate}
			
			\item \textbf{Estudo de modelos de otimização linear}:
				embora a modelagem linear seja bastante restrita, ela pode representar uma grande variedade de problemas de maneira exata ou como aproximação. 
				As restrições desses modelos facilitam sua representação (como sistemas lineares ou matrizes) e oportunizam aos métodos de solução atalhos e generalizações eficientes e bem direcionados.
				
			\item \textbf{Estudo de solução gráfica e método primal simplex}:
				para modelos com somente duas variáveis, a representação bidimensional permite resolução rápida e intuitiva dos problemas de otimização linear, basta identificar o gradiente da função objetivo e a região factível, encontrando a curva de nível (neste caso, o hiperplano perpendicular ao gradiente) no limite da factibilidade no sentido de minimização/maximização. 
				O método simplex aborda o problema considerando que, se houver solução ótima, haverá ao menos um vértice ótimo. Partindo de uma partição básica factível inicial, segue pelas arestas que mais tentem à otimalidade (seja menor ou maior), observando se houver solução e se ela for limitada. 
				
			\item \textbf{Estudo de modelos de otimização inteira}:
				muitos problemas precisam de variáveis binárias ou discretas, geralmente representando escolhas e unidades indivisíveis. 
				Essa maior restrição permite mais adequadamente a modelagem de relações lógicos e funções lineares por partes. 
				
			\item \textbf{Estudo do método \textit{branch-and-bound}}:
				ramificar e limitar descreve o processo de selecionar variáveis para dividir a região factível e buscar limitantes mais precisos.
				São aplicadas múltiplas vezes o método simplex (relaxando linearmente as variáveis inteiras) para obter vértices lineares mais próximos da solução ótima mista, descartando as partes da região em que o limitante não permita haver solução melhor que o melhor limitante conhecido naquele passo.
				
			\item \textbf{Estudo da configuração padrão e dos recursos disponíveis de cada pacote computacional escolhido}: 
				foram escolhidos dois resolvedores comerciais (\solver{CPLEX} \cite{CPLEX} e \solver{Gurobi} \cite{Gurobi}) e dois de código-aberto (\solver{Cbc} \cite{Cbc} e \solver{SCIP} \cite{SCIP}).
				Todos os resolvedores escolhidos aceitam modelos de otimização linear e inteira mista e aplicam de cortes. 
				A interface de linha de comando deles pode ser acessada dentro do ambiente \python{Python} pela biblioteca \python{PuLP} \cite{PuLP}.
				
			\item \textbf{Implementação de um modelo clássico de otimização utilizando os pacotes computacionais escolhidos}:
				o problema de localização de facilidades com capacidade limitada (CFLP) minimizar o custo de transporte e de abertura de facilidades (como armazéns ou fábricas), respeitando suas capacidades pré-estabelecidas. 
				Dados os clientes e as demandas que devem ser satisfeitas, designa-se as facilidades que devem ser abertas, que os devem atender e quanto devem entregar.  
				Como várias facilidades podem entregar produtos para um mesmo cliente, chama-se a versão geral do problema de múltiplas fontes (MS-CFLP).
				A variação de fonte única (SS-CFLP) do problema adiciona a restrição de fornecedores por cliente: somente uma facilidade deve atender cada demanda.
			
			 
			
			\item \textbf{Testes computacionais e avaliação comparativa dos resultados}:
				o \textit{script} desenvolvido lê as instâncias de \textit{benchmark} utilizadas por outros estudos do MS-CFLP e SS-CFLP, monta o modelo misto e o executa em cada resolvedor com diferentes limites de tempo (2, 5 e 10 minutos para instâncias pequenas e 10, 30 e 60 para as grandes).    
				Os resultados registrados nos arquivos de \textit{log} pelos resolvedores \solver{CPLEX}, \solver{Gurobi} e \solver{Cbc} são extraídos com auxílio da biblioteca \python{Orloge} \cite{orloge} (que não oferece suporte para a saída do \solver{SCIP}) e tabelados.  
				
		%		\item \textbf{Estudo de uma técnica de decomposição de um problema prático e o surgimento do modelo escolhido como subproblema}
				
		\end{enumerate}
\fi	

\ifartigo
	\section{Revisão da literatura}
	
\fi		

\ifartigo	
	\section{Definição e modelagem dos problemas} \label{defmodel:cflp}
	
		No problema de localização de facilidades com capacidade limitada (\textit{Capacitated Facility Location Problem}, CFLP) são minimizados os custos de instalação de facilidades e de designação de clientes a elas, de forma a respeitar as limitações de capacidade das facilidades e a satisfazer as demandas dos clientes.      
		Se não forem definidas outras restrições além da capacidade das facilidades e da demanda dos clientes, nada impede que a demanda de um cliente seja satisfeita designando mais do que uma facilidade para ele, o que caracteriza o CFLP com múltiplas fontes (\textit{multi-source}, MS-CFLP) [\ref{defmodel:cflp:ms}].  
		
		Quando cada demanda deve ser atendida por somente uma facilidade, tem-se o CFLP com fonte única (\textit{single-source}, SS-CFLP) [\ref{defmodel:cflp:ss}]. 
		
		Quando houver clientes que não possam ser atendidos por uma mesma facilidade, tem-se o CFLP com incompatibilidade de clientes (\textit{customer incompatibilities}, CFLP-CI) [\ref{defmodel:cflp:ci}].
		 
		\subsection{MS-CFLP: problema de localização de facilidades com capacidade limitada e múltiplas fontes} \label{defmodel:cflp:ms}
		
			São dados o conjunto de facilidades $I$ e o conjunto de clientes $J$.
			Para cada facilidade $i \in I$, tem-se o custo de abertura $f_i \in \mathbb{R}$ e a capacidade máxima de atendimento $s_i \in \mathbb{Z}$. 
			Tem-se também a demanda de produtos $d_j \in \mathbb{Z}$ para cada cliente $j \in J$.
			Para cada par de facilidade e cliente $\langle i, j \rangle \in I \times J$, há o custo $c_{ij} \in \mathbb{R}$ do atendimento de $j$ por $i$ por unidade do produto demandado.  
			
			O modelo pode ser descrito da seguinte forma:
		
			\begin{equation}
				\label{cflp:ms:obj}		
				\min \sum_{i \in I} 
				(
				f_i y_i + \sum_{j \in J} c_{ij} x_{ij}
				)
			\end{equation}
		
			sujeito a 				
			
			\begin{equation}
				\label{cflp:ms:const:demand}		
				\sum_{i \in I} x_{ij} \ge d_j % == 
				\quad
				\forall j \in J
			\end{equation}
		
			\begin{equation}
				\label{cflp:ms:const:capacity}		
				\sum_{j \in J} x_{ij} \le y_i s_i 
				\quad
				\forall i \in I
			\end{equation}
			
			\begin{equation}
				\label{cflp:ms:dom:var}		
				y_i \in \{0, 1\}, 
				\ x_{ij} \ge 0, 
				\ x_{ij} \in \mathbb{Z}
				\quad
				\forall i \in I, \ j \in J
			\end{equation}
		
			O atendimento da demanda do cliente $j \in J$ pela facilidade $i \in I$ é representado pela variável inteira não-negativa $x_{ij}$, cujo valor é a quantidade de unidades do produto são fornecidas por $i$ para $j$.
			A restrição \ref{cflp:ms:const:demand} garante que a soma de todos os atendimentos de cada cliente por todas as facilidades satisfaça sua demanda $d_j$.
			Os custos de atendimento são somados na função objetivo (\ref{cflp:ms:obj}) de acordo com o custo $c_{ij}$ por unidade. 		
			
			A abertura da facilidade $i \in I$ é representada pela variável binária $y_i$, que assume valor 0 quando $i$ estiver fechada e 1 se tiver sido instalada.
			A restrição \ref{cflp:ms:const:capacity} exige que, se a soma dos atendimentos prestados por $i$ para todos os seus clientes for positiva, a facilidade deva ser aberta ($y_i = 1$). 
			Essa restrição também limita a soma dos atendimentos à capacidade $s_i$ da facilidade.
			O custo de instalação $f_i$ da facilidade será somado ao custo na função objetivo (\ref{cflp:ms:obj}) se $y_i = 1$.  
	 
	 		\begin{observation} \label{cflp:ms:obs:int}
	 			É importante ressaltar que esse modelo considera que o produto possui unidades indivisíveis e que, portanto, os atendimentos $x_{ij}$, as capacidades $s_i$ e as demandas $d_j$ são inteiros. 
	 			
	 			Caso fossem fornecidas capacidades reais $s'_i \in \mathbb{R}$, bastaria utilizar $s_i = \lfloor s'_i \rfloor$. 
	 			Se fossem demandas reais $d'_j \in \mathbb{R}$, bastaria utilizar $d_j = \lceil d'_j \rceil$. 
	 			
	 			Contudo, sendo necessário utilizar valores fracionários de atendimento, capacidade e demanda, bastará alterar a restrição \ref{cflp:ms:dom:var} para que sejam reais as variáveis $x_{ij} \in \mathbb{R}$. 
	 			A modelagem do problema funciona para atendimentos no domínio dos reais. 
	 		\end{observation} 		 			
	 		
	 	\subsection{SS-CFLP: problema de localização de facilidades com capacidade limitada e fonte única} \label{defmodel:cflp:ss}		
	 		
	 		A modelagem pode ser essencialmente a mesma do MS-CFLP [\ref{defmodel:cflp:ms}], simplesmente se alterando o domínio de cada $x_{ij} \in \{ 0, d_j \}$ na restrição \ref{cflp:ms:dom:var}. 
	 		Porém a restrição para uma única fonte permite algumas alterações razoáveis.
	 		
	 		Novamente são dados o conjunto de facilidades $I$ e o conjunto de clientes $J$, bem como o custo de abertura $f_i \in \mathbb{R}$ e a capacidade máxima de atendimento $s_i \in \mathbb{Z}$ para cada facilidade $i \in I$. 
	 		Então, para cada par de facilidade e cliente $\langle i, j \rangle \in I \times J$, há a demanda $p_{ij} \in \mathbb{Z}$ (quantas unidades do produto $i$ deveria fornecer a $j$ caso o atenda) e o custo $g_{ij} \in \mathbb{R}$ de atendimento de toda essa demanda.  	 		
	 		A possibilidade de a demanda ser diferente se for atendida por facilidades diferentes é uma generalização razoável permitida pela restrição maior das fontes de atendimento.
	 		
	 		\begin{equation}
	 			\label{cflp:ss:obj}		
	 			\min \sum_{i \in I} 
	 			(
	 			f_i y_i + \sum_{j \in J} g_{ij} x_{ij}
	 			)
	 		\end{equation}
 		
 			sujeito a 				
 			
 			\begin{equation}
 				\label{cflp:ss:const:demand}		
 				\sum_{i \in I} x_{ij} = 1 
 				\quad
 				\forall j \in J
 			\end{equation}
 		
 			\begin{equation}
 				\label{cflp:ss:const:capacity}		
 				\sum_{j \in J} p_{ij} x_{ij} \le y_i s_i 
 				\quad
 				\forall i \in I
 			\end{equation}
 		
 			\begin{equation}
 				\label{cflp:ss:dom:var}		
 				y_i \in \{0, 1\}, 
 				\ x_{ij} \in \{0, 1\}, 
 				\quad
 				\forall i \in I, \ j \in J
 			\end{equation}
 		
 			Dessa vez, $x_{ij}$ é definida como variável binária, representando se a facilidade $i$ atende o cliente $j$. 
 			A restrição \ref{cflp:ss:const:demand} garante que exatamente uma facilidade atenda cada cliente.
 			Os custos de atendimento são somados na função objetivo (\ref{cflp:ss:obj}) de acordo com o custo total $g_{ij}$. 		
 			
 			Novamente, $y_i$ é a variável binária que representa se a facilidade $i \in I$ está aberta.
 			A restrição \ref{cflp:ss:const:capacity} exige que, se a soma dos atendimentos prestados por $i$ para todos os seus clientes for positiva, a facilidade deva ser instalada ($y_i = 1$). 
 			Essa restrição também limita a soma dos atendimentos (considerando a quantidade de produtos $p_{ij}$ de cada um deles) à capacidade $s_i$ da facilidade.
 			Serão somados na função objetivo (\ref{cflp:ss:obj}) os custos de instalação $f_i$ das facilidades abertas ($y_i = 1$).
 			
 			\begin{observation} \label{cflp:ss:obs:int}
 				Assim como o modelo MS-CFLP [\ref{defmodel:cflp:ms}], esse modelo considera indivisibilidade de unidades do produto nas capacidades $s_i$ e demandas $p_{ij}$ inteiras. 
 				Mas se elas fossem reais não haveria nenhuma alteração de funcionamento, dispensando alteração de domínio das variáveis.
 			\end{observation}
 		
 			\begin{observation} \label{cflp:ss:obs:gc} 
 				Embora tenha mudado a representação, não houve alteração nos dados de custo de atendimento. 
 				Pode-se substituir, neste modelo, $g_{ij} = c_{ij} p_{ij}$ e, no MS-CFLP [\ref{defmodel:cflp:ms}], $g_{ij} = c_{ij} d_j$.
 			\end{observation}
 		
 			\begin{observation} \label{cflp:ss:obs:ms}
 				A relaxação linear das variáveis $x_{ij} \in \mathbb{R}$ permitiria a representação de um modelo MS-CFLP [\ref{defmodel:cflp:ms}] com unidades divisíveis.  
 				Contudo, devido à variação da demanda de um mesmo cliente para cada facilidade que o possa atender, esse relaxação modelaria uma situação incomum e estranha. Dessa forma, não utilizaremos instâncias desse formato como se não tivessem a restrição de fonte única.
	 		\end{observation}	
 		
 			\subsubsection{De MS-CFLP para SS-CFLP} \label{defmodel:cflp:ms:ss}
 				Para utilizar dados de uma instância do MS-CFLP [\ref{defmodel:cflp:ms}] e utilizá-la SS-CFLP, basta que, para todo $i \in I$ e $j \in J$, $p_{ij} = d_j$ e $g_{ij} = c_{ij} p_{ij} = c_{ij} d_j$ (conforme a observação \ref{cflp:ss:obs:gc}).
 				
 				Os demais dados são os mesmos para ambos. 
 				
 			    
 				
 		\subsection{MS-CFLP-CI: problema de localização de facilidades com capacidade limitada e incompatibilidade de clientes} \label{defmodel:cflp:ci}	
 		
 			Algumas instâncias que foram utilizadas apresentavam uma restrição a mais.
 			Além da função objetivo (\ref{cflp:ms:obj}), da restrição de atendimento de demanda \ref{cflp:ms:const:demand}, de limitação de capacidade \ref{cflp:ms:const:capacity} e das variáveis definidas na restrição \ref{cflp:ms:dom:var}, é adicionado um conjunto de pares de clientes incompatíveis $\Gamma \subset J^2$. 
 			Os clientes desses pares não podem ser atendidos pela mesma facilidade.
 			
 			\begin{equation}
 				\label{cflp:ms:ci:const:incomp}		
 					x_{ij_1} \le \lambda_{ij_1j_2} s_i, 
 					\ x_{ij_2} \le (1 - \lambda_{ij_1j_2}) s_i 
 					\quad
 					\forall i \in I, \langle j_1, j_2 \rangle \in \Gamma  				
 			\end{equation} 
 		
 			Para isso, são definidas as variáveis de disjunção $\lambda_{ij_1j_2}$
 			
 			\begin{equation}
 				\label{cflp:ms:ci:dom:var}		
 				\lambda_{ij_1j_2} \in \{ 0, 1 \}	
 				\quad
 				\forall i \in I, \langle j_1, j_2 \rangle \in \Gamma	
 			\end{equation}
\fi

		 
	\section{Experimentos computacionais}
	
		Foram escolhidos os resolvedores \solver{Cbc} \cite{Cbc}, \solver{CPLEX} \cite{CPLEX} e \solver{Gurobi} \cite{Gurobi} para comparação de otimalidade, tempo e nós visitados. 
		Os dados para comparação são extraídos dos arquivos de \textit{log} com a biblioteca \python{Orloge} \cite{orloge}. 
		Como ela não oferece suporte para leitura de \textit{log} do \solver{SCIP} \cite{SCIP}, esse resolvedor não foi incluído no estudo.
		
		Foram utilizadas instâncias do CFLP de quatro diferentes conjuntos:
		\begin{enumerate}
			\item \instance{sobolev	} \cite{sobolev}: 	
				esse conjunto de 100 instâncias do modelo SS-CFLP [\ref{defmodel:cflp:ss}] possui, para cada facilidade e cliente, diferentes demandas. 
				\ifartigo
					Foi utilizado somente com fonte única, apresentado na tabela \ref{cflp:tab:1} como \instance{sobolev	.ss}.
				\fi	
				Esse conjunto possui a peculiaridade de alguns pares de facilidade e cliente não terem demanda especificada, indicando que aquela facilidade não pode atender a esse cliente.
				Ele é organizado em 5 grupos de 20 instâncias, tendo em cada um desses a mesma capacidade para todas as facilidades: 10, 20, 30, 40 e 50.
				
			\item \instance{beasley} \cite{beasley}:
				há 36 instâncias do modelo MS-CFLP [\ref{defmodel:cflp:ms}], sendo as primeiras 24 muito pequenas e as 12 últimas, grandes.
				Foi utilizado na forma MS-CFLP original e também com as restrições adicionais de SS-CFLP [\ref{defmodel:cflp:ms:ss}].
				\ifartigo
					Nas tabelas \ref{cflp:tab:1}, \ref{cflp:tab:4} e \ref{cflp:tab:6} é apresentado esse conjunto em partes: 
					\instance{beasley.small.ms} e \instance{beasley.small.ss} referem-se às 24 instâncias menores, com múltiplas fontes (MS) e fonte única (SS);  
					\instance{beasley.large.ms} e \instance{beasley.large.ss} são as 12 grandes, MS e SS.
				\fi	
				
			\item \instance{holmberg} \cite{holmberg}:				
				as 71 instâncias pequenas desse conjunto utilizam o mesmo formato do conjunto \instance{beasley}, também utilizado na forma original MS-CFLP [\ref{defmodel:cflp:ms}] e na adaptada para SS-CFLP [\ref{defmodel:cflp:ms:ss}].
				\ifartigo
					Estão apresentadas nas tabelas \ref{cflp:tab:1} e \ref{cflp:tab:6} como \instance{holmberg.ms} e \instance{holmberg.ss} para as soluções MS e SS.
				\fi	
				
			\item \instance{mess} \cite{mess}:	
				conjunto de 20 instâncias muito grandes no modelo MS-CFLP-CI [\ref{defmodel:cflp:ci}], utilizados como MS-CFLP [\ref{defmodel:cflp:ms}] e SS-CFLP [\ref{defmodel:cflp:ms:ss}].
				\ifartigo
					Chamado de \instance{mess.ms} e \instance{mess.ss} nas tabelas \ref{cflp:tab:5} e \ref{cflp:tab:9}
				\fi	
		\end{enumerate} 
	
		As instâncias são lidas por um \textit{script} \python{Python} \cite{Python} que constrói o modelo de otimização mista da biblioteca \python{PuLP} \cite{PuLP} e executa os resolvedores pela interface de linha de comando.   
		 
		Cada instância de \instance{sobolev	}, \instance{beasley} e \instance{holmberg} foi resolvida três vezes, com os limites de tempo de 2, 5 e 10 minutos.
		As instâncias de \texttt{mess} tiveram limitação maior, de 10, 30 e 60 minutos.
		
		Os testes foram conduzidos em um computador \textit{desktop} Dell XPS 8930 com sistema operacional Ubuntu 22.04.1 LTS 64-bit, processador Intel{\textregistered} Core{\texttrademark} i7-8700 CPU @ 3.20GHz $\times$ 12 e 16 GB de memória RAM.    		
		\ifartigo
			Os resultados estão nas tabelas \ref{cflp:tab:1}, \ref{cflp:tab:4}, \ref{cflp:tab:5}, \ref{cflp:tab:6} e \ref{cflp:tab:9}.
		\fi	
		
		%E um \textit{laptop} Acer Aspire E5-574G com sistema operacional Windows 10 22H2 64-bit, processador Intel{\textregistered} Core{\texttrademark} i7-6500U CPU @ 2.50GHz 2.60GHz e 16 GB de memória RAM.
		
			\includegraphics[width=\textwidth]{res/mess.ss - gap x time.PDF} 
	\includegraphics[width=\textwidth]{res/mess.ss - time x time.PDF} 
	\includegraphics[width=\textwidth]{res/mess.ss - nodes x time.PDF} 
	\includegraphics[width=\textwidth]{res/mess.ss - #Opt x time.PDF} 
	\includegraphics[width=\textwidth]{res/mess.ss - #Fact x time.PDF} 
	\includegraphics[width=\textwidth]{res/mess.ss - #TL x time.PDF} 
	\includegraphics[width=\textwidth]{res/mess.ss - #OfM x time.PDF} 
	\includegraphics[width=\textwidth]{res/sobolev.ss - gap x time.PDF} 
	\includegraphics[width=\textwidth]{res/sobolev.ss - time x time.PDF} 
	\includegraphics[width=\textwidth]{res/sobolev.ss - nodes x time.PDF} 
	\includegraphics[width=\textwidth]{res/sobolev.ss - #Opt x time.PDF} 
	\includegraphics[width=\textwidth]{res/sobolev.ss - #Fact x time.PDF} 
	\includegraphics[width=\textwidth]{res/sobolev.ss - #TL x time.PDF} 
	\includegraphics[width=\textwidth]{res/sobolev.ss - #OfM x time.PDF} 
	\includegraphics[width=\textwidth]{res/holmberg.ss - gap x time.PDF} 
	\includegraphics[width=\textwidth]{res/holmberg.ss - time x time.PDF} 
	\includegraphics[width=\textwidth]{res/holmberg.ss - nodes x time.PDF} 
	\includegraphics[width=\textwidth]{res/holmberg.ss - #Opt x time.PDF} 
	\includegraphics[width=\textwidth]{res/holmberg.ss - #Fact x time.PDF} 
	\includegraphics[width=\textwidth]{res/holmberg.ss - #TL x time.PDF} 
	\includegraphics[width=\textwidth]{res/holmberg.ss - #OfM x time.PDF} 
	\includegraphics[width=\textwidth]{res/beasley.small.ss - gap x time.PDF} 
	\includegraphics[width=\textwidth]{res/beasley.small.ss - time x time.PDF} 
	\includegraphics[width=\textwidth]{res/beasley.small.ss - nodes x time.PDF} 
	\includegraphics[width=\textwidth]{res/beasley.small.ss - #Opt x time.PDF} 
	\includegraphics[width=\textwidth]{res/beasley.small.ss - #Fact x time.PDF} 
	\includegraphics[width=\textwidth]{res/beasley.small.ss - #TL x time.PDF} 
	\includegraphics[width=\textwidth]{res/beasley.small.ss - #OfM x time.PDF} 
	\includegraphics[width=\textwidth]{res/beasley.large.ss - gap x time.PDF} 
	\includegraphics[width=\textwidth]{res/beasley.large.ss - time x time.PDF} 
	\includegraphics[width=\textwidth]{res/beasley.large.ss - nodes x time.PDF} 
	\includegraphics[width=\textwidth]{res/beasley.large.ss - #Opt x time.PDF} 
	\includegraphics[width=\textwidth]{res/beasley.large.ss - #Fact x time.PDF} 
	\includegraphics[width=\textwidth]{res/beasley.large.ss - #TL x time.PDF} 
	\includegraphics[width=\textwidth]{res/beasley.large.ss - #OfM x time.PDF} 
	\includegraphics[width=\textwidth]{res/mess.ms - gap x time.PDF} 
	\includegraphics[width=\textwidth]{res/mess.ms - time x time.PDF} 
	\includegraphics[width=\textwidth]{res/mess.ms - nodes x time.PDF} 
	\includegraphics[width=\textwidth]{res/mess.ms - #Opt x time.PDF} 
	\includegraphics[width=\textwidth]{res/mess.ms - #Fact x time.PDF} 
	\includegraphics[width=\textwidth]{res/mess.ms - #TL x time.PDF} 
	\includegraphics[width=\textwidth]{res/mess.ms - #OfM x time.PDF} 
	\includegraphics[width=\textwidth]{res/holmberg.ms - gap x time.PDF} 
	\includegraphics[width=\textwidth]{res/holmberg.ms - time x time.PDF} 
	\includegraphics[width=\textwidth]{res/holmberg.ms - nodes x time.PDF} 
	\includegraphics[width=\textwidth]{res/holmberg.ms - #Opt x time.PDF} 
	\includegraphics[width=\textwidth]{res/holmberg.ms - #Fact x time.PDF} 
	\includegraphics[width=\textwidth]{res/holmberg.ms - #TL x time.PDF} 
	\includegraphics[width=\textwidth]{res/holmberg.ms - #OfM x time.PDF} 
	\includegraphics[width=\textwidth]{res/beasley.small.ms - gap x time.PDF} 
	\includegraphics[width=\textwidth]{res/beasley.small.ms - time x time.PDF} 
	\includegraphics[width=\textwidth]{res/beasley.small.ms - nodes x time.PDF} 
	\includegraphics[width=\textwidth]{res/beasley.small.ms - #Opt x time.PDF} 
	\includegraphics[width=\textwidth]{res/beasley.small.ms - #Fact x time.PDF} 
	\includegraphics[width=\textwidth]{res/beasley.small.ms - #TL x time.PDF} 
	\includegraphics[width=\textwidth]{res/beasley.small.ms - #OfM x time.PDF} 
	\includegraphics[width=\textwidth]{res/beasley.large.ms - gap x time.PDF} 
	\includegraphics[width=\textwidth]{res/beasley.large.ms - time x time.PDF} 
	\includegraphics[width=\textwidth]{res/beasley.large.ms - nodes x time.PDF} 
	\includegraphics[width=\textwidth]{res/beasley.large.ms - #Opt x time.PDF} 
	\includegraphics[width=\textwidth]{res/beasley.large.ms - #Fact x time.PDF} 
	\includegraphics[width=\textwidth]{res/beasley.large.ms - #TL x time.PDF} 
	\includegraphics[width=\textwidth]{res/beasley.large.ms - #OfM x time.PDF} 


	\vfill
	\section{Referências}
		\bibliographystyle{apalike}
		\bibliography{bibliografia.bib} % https://github.com/akirademenech/MIP_Solvers/blob/main/doc/bibliografia.bib
		
\ifartigo		
	\vfill
	\section{Tabelas}
	\newpage
		\begin{sidewaystable}[]
			
% Please add the following required packages to your document preamble:
% \usepackage{adjustbox}
% \usepackage{multirow}
\begin{table}[]
	\begin{adjustbox}{angle=90}
	\begin{tabular}{cll|lll|lll|lll}
	& & & \multicolumn{3}{c}{\textbf{CBC}} & \multicolumn{3}{c}{\textbf{CPLEX}} & \multicolumn{3}{c}{\textbf{GUROBI}} 	\\\textit{Instâncias} & & & \textbf{120} & \textbf{300} & \textbf{600} & \textbf{120} & \textbf{300} & \textbf{600} & \textbf{120} & \textbf{300} & \textbf{600} \\
\hline
\multirow{7}{*}{\texttt{sobolev.ss}} & \multirow{2}{*}{\textbf{Gap}} & \textbf{Média} & 4,68 & 2,12 & 0,12 & 0 & 0 & 0 & 0,01 & 0 & 0 \\
 & & \textbf{Mediana} & 5,16 & 0 & 0 & 0 & 0 & 0 & 0 & 0 & 0 \\
\cline{2-12}
 & \multirow{2}{*}{\textbf{Time}} & \textbf{Média} & 114,48 & 231,07 & 310,80 & 17,76 & 17,77 & 17,70 & 28,70 & 28,87 & 28,76 \\
 & & \textbf{Mediana} & 119,91 & 275,41 & 274,85 & 14,44 & 14,47 & 14,49 & 19,77 & 19,75 & 19,73 \\
\cline{2-12}
 & \multirow{2}{*}{\textbf{Nodes}} & \textbf{Média} & 4383,73 & 8574,17 & 10707,35 & 12440,43 & 12440,43 & 12440,43 & 20989,02 & 21176,65 & 21176,65 \\
 & & \textbf{Mediana} & 4261 & 9601 & 10410,50 & 9752 & 9752 & 9752 & 14843 & 14843 & 14843 \\
\cline{2-12}
 & \textbf{\#Opt} & & 16 & 57 & 86 & 100 & 100 & 100 & 99 & 100 & 100 \\
 & \textbf{\#Fact} & & 100 & 100 & 100 & 100 & 100 & 100 & 100 & 100 & 100 \\
 & \textbf{\#OfM} & & 0 & 0 & 0 & 0 & 0 & 0 & 0 & 0 & 0 \\
 & \textbf{\#TL} & & 0 & 0 & 0 & 0 & 0 & 0 & 0 & 0 & 0 \\
\hline
\multirow{7}{*}{\texttt{holmberg.ss}} & \multirow{2}{*}{\textbf{Gap}} & \textbf{Média} & 0,03 & 0,03 & 0,02 & 0 & 0 & 0 & 0 & 0 & 0 \\
 & & \textbf{Mediana} & 0 & 0 & 0 & 0 & 0 & 0 & 0 & 0 & 0 \\
\cline{2-12}
 & \multirow{2}{*}{\textbf{Time}} & \textbf{Média} & 7,72 & 12,79 & 19,48 & 0,57 & 0,57 & 0,57 & 0,59 & 0,60 & 0,59 \\
 & & \textbf{Mediana} & 0,36 & 0,36 & 0,35 & 0,14 & 0,14 & 0,14 & 0,13 & 0,13 & 0,13 \\
\cline{2-12}
 & \multirow{2}{*}{\textbf{Nodes}} & \textbf{Média} & 2103,03 & 3901,08 & 5994,70 & 718,46 & 718,46 & 718,46 & 747,58 & 747,58 & 747,58 \\
 & & \textbf{Mediana} & 2 & 2 & 2 & 0 & 0 & 0 & 0 & 0 & 0 \\
\cline{2-12}
 & \textbf{\#Opt} & & 69 & 69 & 70 & 71 & 71 & 71 & 71 & 71 & 71 \\
 & \textbf{\#Fact} & & 71 & 71 & 71 & 71 & 71 & 71 & 71 & 71 & 71 \\
 & \textbf{\#OfM} & & 0 & 0 & 0 & 0 & 0 & 0 & 0 & 0 & 0 \\
 & \textbf{\#TL} & & 0 & 0 & 0 & 0 & 0 & 0 & 0 & 0 & 0 \\
\hline
\multirow{7}{*}{\texttt{beasley.small.ss}} & \multirow{2}{*}{\textbf{Gap}} & \textbf{Média} & 0 & 0 & 0 & 0 & 0 & 0 & 0 & 0 & 0 \\
 & & \textbf{Mediana} & 0 & 0 & 0 & 0 & 0 & 0 & 0 & 0 & 0 \\
\cline{2-12}
 & \multirow{2}{*}{\textbf{Time}} & \textbf{Média} & 0,26 & 0,26 & 0,26 & 0,07 & 0,07 & 0,07 & 0,06 & 0,06 & 0,06 \\
 & & \textbf{Mediana} & 0,17 & 0,17 & 0,17 & 0,04 & 0,04 & 0,04 & 0,04 & 0,04 & 0,04 \\
\cline{2-12}
 & \multirow{2}{*}{\textbf{Nodes}} & \textbf{Média} & 5,83 & 5,83 & 5,83 & 0,62 & 0,62 & 0,62 & 0 & 0 & 0 \\
 & & \textbf{Mediana} & 0 & 0 & 0 & 0 & 0 & 0 & 0 & 0 & 0 \\
\cline{2-12}
 & \textbf{\#Opt} & & 24 & 24 & 24 & 24 & 24 & 24 & 24 & 24 & 24 \\
 & \textbf{\#Fact} & & 24 & 24 & 24 & 24 & 24 & 24 & 24 & 24 & 24 \\
 & \textbf{\#OfM} & & 0 & 0 & 0 & 0 & 0 & 0 & 0 & 0 & 0 \\
 & \textbf{\#TL} & & 0 & 0 & 0 & 0 & 0 & 0 & 0 & 0 & 0 \\
	\end{tabular}
	\end{adjustbox}
	\label{cflp:tab:1}
	\caption{}
\end{table}


		\end{sidewaystable}	
		
		\begin{sidewaystable}[]
			
% Please add the following required packages to your document preamble:
% \usepackage{adjustbox}
% \usepackage{multirow}
\begin{table}[]
	\begin{footnotesize}
	\begin{adjustbox}{angle=90}
	\begin{tabular}{c@{\hskip 0.2cm}l@{\hskip 0.1cm}l|lll|lll|lll}
	& & & \multicolumn{3}{c}{\textbf{CBC}} & \multicolumn{3}{c}{\textbf{CPLEX}} & \multicolumn{3}{c}{\textbf{GUROBI}} 	\\\textit{Instâncias} & & & \textbf{120} & \textbf{300} & \textbf{600} & \textbf{120} & \textbf{300} & \textbf{600} & \textbf{120} & \textbf{300} & \textbf{600} \\
\hline
\multirow{7}{*}{\texttt{beasley.large.ss}} & \multirow{2}{*}{\textbf{Gap}} & \textbf{Média} & 80,44 & 80,05 & 55,24 & 4,24 & 1,01 & 0,58 & 5,36 & 1,13 & 0,53 \\
 & & \textbf{Mediana} & 78,50 & 78,06 & 77,94 & 3,13 & 0 & 0 & 5,27 & 0,41 & 0 \\
\cline{2-12}
 & \multirow{2}{*}{\textbf{Time}} & \textbf{Média} & 129,87 & 313,05 & 609,31 & 120,05 & 241,57 & 322,60 & 120,11 & 292,33 & 446,46 \\
 & & \textbf{Mediana} & 128,53 & 308,43 & 601,07 & 120,04 & 256,37 & 252,09 & 120,02 & 300,01 & 452,43 \\
\cline{2-12}
 & \multirow{2}{*}{\textbf{Nodes}} & \textbf{Média} & 0 & 1,58 & 128,83 & 77,58 & 1376,33 & 2211,25 & 1279,75 & 1858,75 & 3475,42 \\
 & & \textbf{Mediana} & 0 & 0 & 4,50 & 0 & 1079,50 & 1886 & 1284 & 1367 & 2813,50 \\
\cline{2-12}
 & \textbf{\#Opt} & & 0 & 0 & 0 & 0 & 7 & 9 & 0 & 3 & 9 \\
 & \textbf{\#Fact} & & 12 & 12 & 12 & 12 & 12 & 12 & 12 & 12 & 12 \\
 & \textbf{\#OfM} & & 0 & 0 & 0 & 0 & 0 & 0 & 0 & 0 & 0 \\
 & \textbf{\#TL} & & 0 & 0 & 0 & 0 & 0 & 0 & 0 & 0 & 0 \\
	\end{tabular}
	\end{adjustbox}
	\end{footnotesize}
	\caption{}
	\label{cflp:tab:4}
\end{table}


			
% Please add the following required packages to your document preamble:
% \usepackage{adjustbox}
% \usepackage{multirow}
\begin{table}[]
	\begin{footnotesize}
	\begin{adjustbox}{angle=90}
	\begin{tabular}{c@{\hskip 0.2cm}l@{\hskip 0.1cm}l|lll|lll|lll}
	& & & \multicolumn{3}{c}{\textbf{CBC}} & \multicolumn{3}{c}{\textbf{CPLEX}} & \multicolumn{3}{c}{\textbf{GUROBI}} 	\\\textit{Instâncias} & & & \textbf{600} & \textbf{1800} & \textbf{3600} & \textbf{600} & \textbf{1800} & \textbf{3600} & \textbf{600} & \textbf{1800} & \textbf{3600} \\
\hline
\multirow{7}{*}{\texttt{mess.ss}} & \multirow{2}{*}{\textbf{Gap}} & \textbf{Média} & 19,53 & 17,34 & 15,14 & 2,40 & 1,32 & 0,92 & 1,72 & 1,25 & 0,82 \\
 & & \textbf{Mediana} & 29,47 & 21,24 & 9,41 & 1,99 & 0,93 & 0,59 & 1,46 & 1,07 & 0,73 \\
\cline{2-12}
 & \multirow{2}{*}{\textbf{Time}} & \textbf{Média} & 509,77 & 1517,66 & 2978,46 & 481,98 & 1333,36 & 2553,62 & 486,70 & 1410,30 & 2648,31 \\
 & & \textbf{Mediana} & 597,25 & 1794,17 & 3571,55 & 600,21 & 1800,38 & 3601 & 600,05 & 1800,06 & 3600,04 \\
\cline{2-12}
 & \multirow{2}{*}{\textbf{Nodes}} & \textbf{Média} & 4341 & 22587,15 & 65098,17 & 31113,92 & 75832,91 & 154435,20 & 29797,69 & 67568,38 & 140780,45 \\
 & & \textbf{Mediana} & 672 & 2127 & 10508 & 540 & 8167 & 39436 & 1813 & 6608 & 15406 \\
\cline{2-12}
 & \textbf{\#Opt} & & 2 & 2 & 2 & 3 & 3 & 3 & 3 & 3 & 3 \\
 & \textbf{\#Fact} & & 7 & 7 & 7 & 13 & 11 & 10 & 13 & 13 & 11 \\
 & \textbf{\#OfM} & & 0 & 0 & 0 & 0 & 2 & 3 & 0 & 0 & 1 \\
 & \textbf{\#TL} & & 6 & 6 & 5 & 0 & 0 & 0 & 0 & 0 & 0 \\
	\end{tabular}
	\end{adjustbox}
	\end{footnotesize}
	\caption{}
	\label{cflp:tab:5}
\end{table}


			
% Please add the following required packages to your document preamble:
% \usepackage{adjustbox}
% \usepackage{multirow}
\begin{table}[]
	\begin{adjustbox}{angle=90}
	\begin{tabular}{cll|lll|lll|lll}
	& & & \multicolumn{3}{c}{\textbf{CBC}} & \multicolumn{3}{c}{\textbf{CPLEX}} & \multicolumn{3}{c}{\textbf{GUROBI}} 	\\\textit{Instâncias} & & & \textbf{600} & \textbf{1800} & \textbf{3600} & \textbf{600} & \textbf{1800} & \textbf{3600} & \textbf{600} & \textbf{1800} & \textbf{3600} \\
\hline
\multirow{7}{*}{\texttt{mess.ms}} & \multirow{2}{*}{\textbf{Gap}} & \textbf{Média} & 10,16 & 8,55 & 6,16 & 0,13 & 0,07 & 0,03 & 0,11 & 0,04 & 0,02 \\
 & & \textbf{Mediana} & 2,57 & 2,07 & 2,19 & 0 & 0 & 0 & 0,09 & 0 & 0 \\
\cline{2-12}
 & \multirow{2}{*}{\textbf{Time}} & \textbf{Média} & 490,95 & 1439,28 & 2625,29 & 279,55 & 731,79 & 985,22 & 356,01 & 883,96 & 1104,66 \\
 & & \textbf{Mediana} & 591 & 1762,92 & 3437,64 & 209,53 & 210,01 & 45,26 & 600,10 & 810,70 & 152,61 \\
\cline{2-12}
 & \multirow{2}{*}{\textbf{Nodes}} & \textbf{Média} & 4285,33 & 10993,40 & 22013,58 & 10715,88 & 25935,38 & 44222,29 & 7694,91 & 21289,90 & 24029,50 \\
 & & \textbf{Mediana} & 1176 & 4239 & 3582,50 & 7133,50 & 15520,50 & 6271 & 2675 & 10121,50 & 14357 \\
\cline{2-12}
 & \textbf{\#Opt} & & 3 & 3 & 4 & 5 & 5 & 6 & 5 & 6 & 6 \\
 & \textbf{\#Fact} & & 15 & 15 & 12 & 8 & 8 & 7 & 11 & 10 & 8 \\
 & \textbf{\#OfM} & & 1 & 0 & 3 & 0 & 0 & 1 & 3 & 3 & 5 \\
 & \textbf{\#TL} & & 0 & 0 & 0 & 0 & 0 & 0 & 0 & 0 & 0 \\
	\end{tabular}
	\end{adjustbox}
	\label{cflp:tab:0}
	\caption{}
\end{table}


		\end{sidewaystable}	
		
		\begin{sidewaystable}[]	
			
% Please add the following required packages to your document preamble:
% \usepackage{adjustbox}
% \usepackage{multirow}
\begin{table}[]
	\begin{adjustbox}{angle=90}
	\begin{tabular}{cll|lll|lll|lll}
	& & & \multicolumn{3}{c}{\textbf{CBC}} & \multicolumn{3}{c}{\textbf{CPLEX}} & \multicolumn{3}{c}{\textbf{GUROBI}} 	\\\textit{Instâncias} & & & \textbf{120} & \textbf{300} & \textbf{600} & \textbf{120} & \textbf{300} & \textbf{600} & \textbf{120} & \textbf{300} & \textbf{600} \\
\hline
\multirow{7}{*}{\texttt{holmberg.ms}} & \multirow{2}{*}{\textbf{Gap}} & \textbf{Média} & 0,03 & 0,01 & 0 & 0 & 0 & 0 & 0 & 0 & 0 \\
 & & \textbf{Mediana} & 0 & 0 & 0 & 0 & 0 & 0 & 0 & 0 & 0 \\
\cline{2-12}
 & \multirow{2}{*}{\textbf{Time}} & \textbf{Média} & 3,69 & 6,19 & 8,49 & 0,25 & 0,26 & 0,25 & 0,17 & 0,17 & 0,17 \\
 & & \textbf{Mediana} & 0,36 & 0,33 & 0,34 & 0,13 & 0,14 & 0,12 & 0,11 & 0,11 & 0,11 \\
\cline{2-12}
 & \multirow{2}{*}{\textbf{Nodes}} & \textbf{Média} & 188,49 & 354,62 & 583,15 & 72,42 & 72,42 & 72,42 & 10,76 & 10,76 & 10,76 \\
 & & \textbf{Mediana} & 2 & 2 & 2 & 0 & 0 & 0 & 0 & 0 & 0 \\
\cline{2-12}
 & \textbf{\#Opt} & & 70 & 70 & 71 & 71 & 71 & 71 & 71 & 71 & 71 \\
 & \textbf{\#Fact} & & 71 & 71 & 71 & 71 & 71 & 71 & 71 & 71 & 71 \\
 & \textbf{\#OfM} & & 0 & 0 & 0 & 0 & 0 & 0 & 0 & 0 & 0 \\
 & \textbf{\#TL} & & 0 & 0 & 0 & 0 & 0 & 0 & 0 & 0 & 0 \\
\hline
\multirow{7}{*}{\texttt{beasley.small.ms}} & \multirow{2}{*}{\textbf{Gap}} & \textbf{Média} & 0 & 0 & 0 & 0 & 0 & 0 & 0 & 0 & 0 \\
 & & \textbf{Mediana} & 0 & 0 & 0 & 0 & 0 & 0 & 0 & 0 & 0 \\
\cline{2-12}
 & \multirow{2}{*}{\textbf{Time}} & \textbf{Média} & 0,39 & 0,39 & 0,39 & 0,08 & 0,08 & 0,08 & 0,07 & 0,07 & 0,07 \\
 & & \textbf{Mediana} & 0,11 & 0,11 & 0,11 & 0,06 & 0,06 & 0,06 & 0,04 & 0,04 & 0,04 \\
\cline{2-12}
 & \multirow{2}{*}{\textbf{Nodes}} & \textbf{Média} & 45,17 & 45,17 & 45,17 & 11,46 & 11,46 & 11,46 & 0 & 0 & 0 \\
 & & \textbf{Mediana} & 6 & 6 & 6 & 0 & 0 & 0 & 0 & 0 & 0 \\
\cline{2-12}
 & \textbf{\#Opt} & & 24 & 24 & 24 & 24 & 24 & 24 & 24 & 24 & 24 \\
 & \textbf{\#Fact} & & 24 & 24 & 24 & 24 & 24 & 24 & 24 & 24 & 24 \\
 & \textbf{\#OfM} & & 0 & 0 & 0 & 0 & 0 & 0 & 0 & 0 & 0 \\
 & \textbf{\#TL} & & 0 & 0 & 0 & 0 & 0 & 0 & 0 & 0 & 0 \\
\hline
\multirow{7}{*}{\texttt{beasley.large.ms}} & \multirow{2}{*}{\textbf{Gap}} & \textbf{Média} & 82,74 & 82,45 & 69,94 & 1,92 & 0,59 & 0,42 & 5,31 & 1,70 & 0,31 \\
 & & \textbf{Mediana} & 80,94 & 80,61 & 80,50 & 0,32 & 0 & 0 & 4,01 & 0,75 & 0 \\
\cline{2-12}
 & \multirow{2}{*}{\textbf{Time}} & \textbf{Média} & 130,41 & 308,56 & 607,54 & 106,23 & 183,49 & 258,48 & 111,08 & 261,11 & 392,52 \\
 & & \textbf{Mediana} & 128,40 & 309,72 & 603,09 & 120,07 & 183,93 & 183,28 & 120,02 & 300,03 & 371,93 \\
\cline{2-12}
 & \multirow{2}{*}{\textbf{Nodes}} & \textbf{Média} & 0 & 0,17 & 22,33 & 580,83 & 2314,75 & 3291,58 & 1043,17 & 1182,50 & 1641,83 \\
 & & \textbf{Mediana} & 0 & 0 & 0 & 546,50 & 2847 & 2847 & 1262 & 1312,50 & 1546,50 \\
\cline{2-12}
 & \textbf{\#Opt} & & 0 & 0 & 0 & 5 & 9 & 9 & 2 & 2 & 10 \\
 & \textbf{\#Fact} & & 12 & 12 & 12 & 12 & 12 & 12 & 12 & 12 & 12 \\
 & \textbf{\#OfM} & & 0 & 0 & 0 & 0 & 0 & 0 & 0 & 0 & 0 \\
 & \textbf{\#TL} & & 0 & 0 & 0 & 0 & 0 & 0 & 0 & 0 & 0 \\
	\end{tabular}
	\end{adjustbox}
\end{table}


		\end{sidewaystable}		
		
\fi		

\end{document}
